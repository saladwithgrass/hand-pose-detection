% !TEX root = ./main.tex

\documentclass[12pt, a4paper]{article}
\usepackage{graphicx} % Required for inserting images
\usepackage[utf8]{inputenc}
\usepackage[english, russian]{babel}
\usepackage{times}
\usepackage{fontspec} 
\defaultfontfeatures{Ligatures={TeX},Renderer=Basic} 
\setmainfont[Ligatures={TeX,Historic}]{Times New Roman}

\usepackage{biblatex}
\addbibresource{sources.bib}

\usepackage{indentfirst}
\setlength\parindent{24pt}

\title{Система визуальной телеоперации манипулятором}
\author{Скоробогатов Егор}
\date{Ноябрь 2024}

\begin{document}

\maketitle

\section{Введение}
    Для телеоперации зачастую нужно дорогостоящее оборудование, 
    физическая модель и дорогостоящие датчики. 
    В робототехнике упрощение телеоперации может помочь с обучением 
    embodied AI моделей без нужды в дополнительном оборудовании. 
    На данный момент есть способы, позволяющие собирать данные для роботов с минимумом оборудования \cite{umi},
    но многие из них требуют предварительной подготовки и хороших камер. 
    В этой статье я рассмотрю способ получения положения ключевых 
    точек руки на основе двух камер
    и нахождение положения и ориентации захвата при помощи двух RGB камер,
    которые стоят гораздо дешевле любого упомянутого выше оборудования.


\section{Анализ других работ}
    \subsection{Обнаружение ключевых точек}
        Во многих работах рассматривается обнаружение КТР основываясь 
        на одном RGB изображении \cite{interhand} 
        \cite{multiviewbootstrapping}.
        
    \subsection{Нахождение положения}
    \subsection{Приведение к захвату}
    но зачастую в прикладной робототехнике, на данный момент, используются 
    захваты не с пятью пальцами, а с двумя или тремя, 
    что усложняет применимость этих данных для обучения роботов. 
    Подход для снятия движений манипулятора, который я попытаюсь 
    реализовать похож на описанный в работе \cite{anyteleop},
    но с тем отличием, что я попытаюсь имплементировать приведение руки к 
    захвату с 2-мя пальцами. 

\section{}
\printbibliography[]

\end{document}
