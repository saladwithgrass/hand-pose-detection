% !TEX root = ./main.tex

\documentclass[12pt, a4paper]{article}
\usepackage{graphicx} % Required for inserting images
\usepackage[utf8]{inputenc}
\usepackage[english, russian]{babel}
\usepackage{times}
\usepackage{fontspec} 
\usepackage{amsmath}
\defaultfontfeatures{Ligatures={TeX},Renderer=Basic} 
\setmainfont[Ligatures={TeX,Historic}]{Times New Roman}

\usepackage{biblatex}
\addbibresource{sources.bib}

\usepackage{indentfirst}
\setlength\parindent{12.5mm}

\usepackage{geometry}
\geometry{
    a4paper,
    left=30mm,
    right=10mm,
    top=20mm,
    bottom=20mm
}

\usepackage{setspace}
\onehalfspacing

\title{Система визуальной телеоперации манипулятором}
\author{Скоробогатов Егор}
\date{Ноябрь 2024}

\begin{document}
\onehalfspacing

\maketitle

\section{Введение}

    С развитием машинного обучения и его применения в робототехнике, выросла нужда в записи данных для обучения так называемых "embodied AI". 
    Также очень полезной практикой является запись данных в реальных и разных условиях.

    В данной работе представлена система для определения трехмерного положения кисти в пространстве с использованием двух камер.
    Эта система может использоваться для записи датасетов для обучения роботов построению траекторий, управлению мехатронными телами, а также для простого запоминания повторяемых действий.
    Система включает в себя модули калибровки камер, распознавания рук и триангуляции, что позволяет точно определять координаты ключевых точек рук в реальном времени.

\section{Обзор системы}

Система состоит из нескольких ключевых модулей:
\begin{enumerate}
    \item Калибровка камер
    \item Распознавание рук
    \item Триангуляция
    \item Утилиты
\end{enumerate}

\section{Модуль калибровки камер}

Модуль калибровки камер отвечает за определение внутренних параметров каждой камеры, таких как фокусное расстояние и дисторсия \cite{zhang2000flexible}. Этот модуль использует шахматную доску Charuco для калибровки каждой камеры в отдельности (\texttt{calibration/intrinsics\_calibration.py}). Также, модуль содержит функциональность для определения ориентации камер относительно друг друга (\texttt{calibration/orientation\_calibration.py}). Это необходимо для точной триангуляции. Для сохранения параметров калибровки используется формат \texttt{.pkl}, в который можно добавить размеры изображения (\texttt{calibration/calibration\_data/add\_image\_size.py}).
\textit{Модуль калибровки крайне важен для обеспечения точности измерений!}

\textit{Пример цитирования:} \cite{bradski2000opencv}

\section{Модуль распознавания рук}

Модуль распознавания рук использует библиотеку MediaPipe \cite{lugaresi2019mediapipe} для обнаружения и определения ключевых точек рук на изображениях с каждой камеры (\texttt{detection/capture\_detector.py}). Модуль принимает видеопоток с камеры и возвращает координаты ключевых точек рук в пикселях. Для повышения производительности используется многопоточная обработка кадров (\texttt{detection/run\_detection.py}).

\textit{Пример цитирования:} \cite{mediapipe}

\section{Модуль триангуляции}

Модуль триангуляции использует координаты ключевых точек рук с каждой камеры и параметры калибровки для определения трехмерных координат этих точек в пространстве (\texttt{positioning/triangulator.py}). Модуль использует метод DLT (Direct Linear Transformation) для триангуляции \cite{hartley2003multiple}. Также, модуль содержит функциональность для отображения результатов триангуляции в 3D (\texttt{utils/visualizer\_3d.py}). Для повышения точности, модуль использует параметры ориентации камер, полученные в модуле калибровки (\texttt{positioning/run\_hand\_positioning.py}, \texttt{positioning/run\_aruco\_positioning.py}).

\textit{Пример цитирования:} \cite{faugeras1993three}

\section{Утилиты}

Вспомогательные функции для работы с файлами, визуализации и общими задачами. К ним относятся:
\begin{itemize}
    \item \texttt{utils/file\_utils.py}: Функции для работы с файлами и папками.
    \item \texttt{utils/visualizer\_3d.py}: Класс для визуализации трехмерных данных.
    \item \texttt{utils/general\_utils.py}: Общие утилиты, такие как запросы подтверждения.
    \item \texttt{utils/draw\_utils.py}: Функции для рисования на изображениях.
\end{itemize}

\section{Заключение}

В данной работе была представлена система для определения трехмерного положения рук с использованием нескольких камер. Система состоит из модулей калибровки камер, распознавания рук и триангуляции. Результаты работы могут быть использованы в различных областях, таких как робототехника, виртуальная реальность и взаимодействие человека с компьютером.
\printbibliography
% \bibliographystyle{plain}
% \bibliography{sources}

\end{document}